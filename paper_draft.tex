%% Author:  Alexander A. Massoud
%% Date:    7/20/2021

\documentclass[12pt, letterpaper]{book}
\usepackage[english]{babel}
\usepackage[utf8]{inputenc}

\pagestyle{headings}

\begin{document}

\chapter{Introduction}
Subsequent planet formation following star formation can be described using the chemical composition and evolution of gas and dust in dense molecular clouds. The behavior of gas and dust when exposed to ultraviolet (UV) radiation is dependent on the metallicity of the host galaxy. Changes in the behavior of gas and dust regulate star formation, dust grain and molecule collisions, and the ionization of the interstellar medium (ISM). Metal-poor galaxies resemble an earlier epoch of galactic evolution because heavy element abundances in a galaxy increase as stars evolve and lose their mass to the ISM. The lower metallicity may have measurable effects on abundances of molecular coolants, UV and collisional excitation and de-excitation, the dust-to-gas ratio, and the timescale for planet formation. Metal-poor galaxies are therefore interesting regions for identifying and characterizing circumstellar disks and other highly structured precursors to planet formation.

The mechanisms by which hydrogen molecules change quantum energy levels make them a powerful probe for the physics of interstellar environments. Molecular hydrogen ($\textrm{H}_2$) is the most abundant molecule in the Universe. Only three mechanisms can excite it: inelastic collisions (thermal), X-rays, and far-ultraviolet radiation (FUV). The formation of $\textrm{H}_2$ on interstellar dust grains can also contribute to its excitation, but this contribution is not dominant.

The method of excitation for $\textrm{H}_2$ can be described as a combination of these three mechanisms, with collisions and FUV being the most influential. The $\textrm{H}_2$ rotational ({\em J}) and vibrational ({\em v}) levels, referred to as ``rovibrational'' levels for the rest of this paper, have distinct population distributions depending on which of these mechanisms dominate. In dense and/or hot gas, the rovibrational levels are excited and de-excited in shock fronts of numerous collisions and give rise to a thermal distribution. In low-density cool gas surrounding a high mass star, UV irradiation excites the rovibrational levels. The level population distribution is distinctly non-thermal in this case. It does not decline monotonically as the excitation energy increases and is distinguished by a distinct ``sawtooth'' shape when plotted. However, these are two limiting cases for the rovibrational level populations. Sources tend to show a mixture of thermally excited and UV excited level populations.

Different regions in young star forming regions can still be distinguished using observations of the rovibrational line flux ratios. $\textrm{H}_2$ has been used as a line flux diagnostic to characterize gas in the ISM, planetary nebulae (PNe), and circumstellar environments (e.g. Shull and Beckwith (1982), Lumsden et al.(2001), Martin-Za{\"\i}di et al. (2007, 2008)). Most of the ISM in galaxies or PNe with massive star formation is comprised of photo-dissociation regions (PDRs). PDRs are highly structured interfaces around stars where the ISM transitions from hot ionized to cool molecular gas. The differentiated structure of a PDR is defined by the transitions of hydrogen from ionized (HII), to neutral atomic (HI), and then to molecular ($\textrm{H}_2$) states. FUV photons with energies $11.2$ to $13.6$ eV, $912 < \lambda < 1110$ {\AA} (just below the Lyman continuum), pass through the ionization front between HII and HI but are reduced by dust, elements with lower ionization potentials, and the Lyman and Werner bands of $\textrm{H}_2$ in the dissociation front between HI and $\textrm{H}_2$. The remaining FUV photons pass beyond the dissociation front and are rapidly absorbed by the cold molecular gas outside of the PDR.

PDRs are found in star forming regions where clouds of molecular gas surround stars typically ten to fifteen times as massive as the Sun. The PDRs around these high mass stars are therefore only observed in young star forming regions or regions of massive star death. At a mean distance of $61.7$ kpc from the Sun, the Small Magellanic Cloud (SMC) is a relatively close galactic neighbor with high star formation activity. The SMC is a dwarf irregular galaxy that has lower abundances of heavy metals than the Milky Way (one-fifth solar metallicity or $Z = 0.004$), relatively low internal extinction, and a relatively low overall abundance of polycyclic aromatic hydrocarbons (PAHs) and dust in its ISM. Dwarf irregular galaxies like the SMC are metal-poor relative to the Milky Way and still harbor massive star formation, making them useful for targeted studies of planet formation around young massive stars. Descriptions of cold molecular gas with chemistry conducive to future planet formation in the SMC would allow us to characterize planet formation during early epochs of the Universe.

Protoplanetary disk candidates have been observed in the SMC beginning when Galley et al. (1982) made the first identification of a protostar there. However, very few have follow-up observations that constrain the candidates to be Herbig Ae/Be (HAeBe) stars. HAeBe stars are the most massive stars (2-10 $M_{\odot}$) approaching the main sequence without dense circumstellar envelopes that obscure their photospheres and circumstellar disks. The protoplanetary disks around HAeBe stars in the SMC may evolve with less irradiation shielding from dust due to the galaxy's low metallicity. We need a large sample of protoplanetary disk candidates around HAeBe stars in the SMC to characterize the unique processes that drive planet formation there. To this end, Keller et al. (2019) conducted a multi-wavelength survey of point sources in the SMC and identified 13 candidate HaeBe stars, 3 of which have $\textrm{H}_2$ line emission.

We have identified 22, 14, and 21 strong rovibrational $\textrm{H}_2$ emission lines in each of our point sources, respectively. In Section \ref{chapter:observations}, we describe our near-infrared spectra of the point sources taken with the TripleSpec 4.1 (referred to in this paper as ARCoIRIS) imaging spectrograph. In Section \ref{chapter:data_reduction}, we describe the data reduction, spectra extraction and combination, atmospheric correction, continuum fit and subtraction, wavelength calibration, and method for extracting line fluxes. We discuss how we calculate the column density from each line flux and generate an excitation diagram for each point source in Section \ref{chapter:analysis}. In Section \ref{chapter:model}, we compare the observed $H_2$ rovibrational level populations to those predicted by Cloudy models to check whether each point source resembles a PDR, PNe, or protoplantary disk, determine where the observed level populations deviate from predictions for metal-rich environments, see if we can correct those deviations, and determine which model provides the best fit to each point source. We summarize and discuss our conclusions in Section 6.

\chapter{Observations}
\label{chapter:observations}

The data were taken with the ARCoIRIS on the $4.0$-m Victor Blanco Telescope at the Cerro Tololo Inter-American Observatory on the nights of November 2016 [INSERT DATES]. For future reference, ARCoIRIS was moved to the $4.1$-m Southern Astrophysical Research Telescope at the National Optical Astronomy Observatory during the 2019A semester. ARCoIRIS is an infrared cross-dispersed imaging spectrograph that covers a simultaneous wavelength range of $0.80$ to $2.47$ $\mu$m at a high spectral resolution of $R = \lambda/\Delta\lambda \approx 3500$. Long-exposure and flatfield calibration frames were taken through the ARCoIRIS control software at the beginning of each night. ARCoIRIS features a fixed slit that subtends $1.1'' \times 28''$ on the sky when the instrument is mounted on the $4.0$-m telescope at Cerro Tololo Inter-American Observatory. The spectrum of our point sources are not spatially extended and do not fill the whole slit. We took minute exposures on the targets and minute exposures on the sky. The sky exposures were used to subtract atmospheric (telluric) contamination.

\section{Details of Source Selection}
 Keller et al. (2019) described the collection of mid-infrared spectra, optical spectra, and archival photometry spanning the optical to far-infrared for 17 infrared point sources. They selected these 17 objects for further observation from a large sample of infrared point sources because the objects have $H\alpha$ emission. We have extracted $\textrm{H}_2$ line flux from 3 of these 17 objects, referred to as lk250, lk215, and lk210 throughout the rest of this paper.

\section{Near-Infrared Spectra}
We used ARCoIRIS to obtain near-infrared spectra in the wavelength range $0.80$-$2.47$ $\mu$m. The spectra are divided into six spectral orders which are projected onto the science detector. In photometric terms, the spectral orders encompass the entire z'YJHK range. Our data reduction pipeline, described in detail in Chapter \ref{chapter:data_reduction}, worked on the H and K bands, spanning $1.143$-$2.467$ $\mu$m. We rejected the other orders to focus our analysis on the NIR $\textrm{H}_2$ lines listed in Table 1.

\chapter{Data Reduction and Line Flux Extraction}
\label{chapter:data_reduction}

\section{Reducing Telescope Data}
To reduce the ARCoIRIS data, we used the {\em Spextool} data reduction package. {\em Spextool} was intended to reduce spectral data obtained with SpeX, an instrument at the NASA Infrared Telescope Facility, and performs a suite of procedures to produce fully reduced spectra [Cushing]. We ran {\em Spextool} on our ARCoIRIS data using the methods described by Dr.\ Katelyn Allers of Bucknell University in her four-part YouTube tutorial series. I created a master flat by combining flatfield images, and {\em Spextool} measured the difference in pixel-to-pixel sensitivity and uneven illumination on the detector array using the master flat. I created a wavelength calibration image by combining long-exposure sky frames. {\em Spextool} uses OH emission lines from the sky frames and emission from an argon discharge lamp to estimate the wavelength solution. The final wavelength solution corrects the pixel sensitivity and wavelength precision of the data.

\section{Extracting and Combining Spectra}
The spectrum of the region, which is not spatially extended and does not fill the whole slit, is extracted by subtracting the sky frames from the science frames. The extracted spectrum is calibrated for pixel sensitivity and wavelength precision and only contains light from the region. I extracted a spectrum for each observation-run. I identified and masked noisy pixels. The spectra were combined into a master spectrum.

\section{Telluric Correction}
Reliable identification of H$_2$ rovibrational line emission is complicated by molecules in the Earth's atmosphere that are active primarily in the NIR. To obtain precise measurements of relative column densities of H$_2$ rovibrational states, we need to correct for telluric contamination. We ran the {\em Spextool} telluric correction routine that assumes our spectra has a continuum shape similar to a photometric standard star. {\em Spextool} then constructs a convolution kernel using the hydrogen absorption feature at approximately 1.0 $\mu$m that masks telluric absorption lines. {\em Spextool} fits the photometric standard star to the features in our master spectrum and removes the telluric features once the fit is successful. This new master spectrum is wavelength calibrated, extracted, combined and telluric corrected.

\section{Continuum Fit and Subtraction}
The faint continuum is the result of dust grain emission along with emission from ionized gas. The continuum is subtracted from each order using Python package \texttt{specutils}. \texttt{specutils} is a Python package with built-in routines for containing, manipulating, and analyzing spectroscopic data. For the rest of this paper, we carry out our analysis using Python routines using a combination of \texttt{specutils} and Astropy.
\texttt{specutils} smooths the master spectrum using a median filtering technique. The median continuum is then subtracted from the master spectrum. We did not include regions with a large amount of remnant telluric absorption after correction in our input spectrum. This technique fits the continuum without subtracting narrow features in the spectrum, such as emission lines. We analyzed each order as a separate 2D spectrum.

\section{Line Wavelengths}
The spectral resolution of ARCoIRIS is large enough that we can resolve lines $< 2 \times 10^{-3}$ $\mu$m apart. To correct for the redshift of the SMC, the spectra were shifted by $12 \times 10^{-4}$ $\mu$m before being compared to the vacuum wavelengths for the H$_2$ lines. The H$_2$ vacuum wavelengths were calculated by Kaplan et. al (2017) from the theoretical ground electronic state rovibrational energy levels given in Komasa et al (2011). We chose their work as a reference for our own because the Orion Bar is another duty photo-dissociated region with a central star that is the source of UV radiation. We reproduce the calculated vacuum wavelengths in column 1 of Table 1. We report our measured difference between the observed line centroids and the theoretical wavelengths as $\Delta\lambda$ in column 2 of Table 1.

We used our continuum-subtracted master spectrum to generate a new spectrum with a constant uncertainty. \texttt{specutils} finds emission and absorption lines in the spectrum based on deviations larger than a noise factor that is a scaled multiple of the constant uncertainty. We built a table of observed H$_2$ line emission from emission lines that fell within the resolving power of the theoretical wavelengths.

We visually inspected each observed emission line to ensure that it is a real signal. Lines that appeared to be contaminated by blends, telluric features, or noise spikes were rejected. We measured the noise of each real feature as the standard deviation of the continuum to the right and left of the feature. The S/N ratio of each feature is reported in column 5 of Table 1.

\section{Line Flux Extraction}
We extract line fluxes by fitting all the H$_2$ rovibrational transition lines we observe in the SMC with a one dimensional Gaussian model, which we integrate to obtain a direct measure of the line flux. Figure 1 compares the master spectrum of several of the lines we observed to the fit of the emission component and depicts the residual between the observed spectrum and the fit model.

\chapter{Analysis}
\label{chapter:analysis}

\section{Calculating $\textrm{H}_2$ Level Populations}
We calculate the column density of $\textrm{H}_2$ in the upper $J$ energy state

\begin{equation} \label{eq:col_dense}
	N_J = \frac{F_J}{\Delta E_J h c A_J}
\end{equation}

\noindent where $F_J$ is the line flux of the transition from upper to lower rovibrational states in erg s$^{-1}$ cm$^{-2}$, $\Delta E_J$ is the difference in energy between the states in wavenumbers (cm$^{-1}$), and $A_J$ is the transition probability (s$^{-1}$) (we follow the value selection from Kaplan et al. 2017 and use the values from Wolniewicz et al. 1998, which are the values used by Cloudy). $h$ and $c$ represent the speed of light and Planck's constant, respectively.

We measured relative fluxes for 35 lines in lk250, 22 lines in lk215, and 30 lines in lk210. All lines we measured had S/N $>$ 4, shown in column 5 of Table 1. We normalize the relative $N_J$ values reported in Table 1 to the flux of the 4-2 O(3) line, which we label as the reference flux, $N_r$. We followed the analysis method of Kaplan et al. 2017 and chose the 4-2 O(3) line as our reference because it is bright in each of our point sources and it is primarily excited by UV photons. Identified transitions from the same upper level (e.g. 1-0 S(1) and 1-0 Q(3)) provide multiple independent measurements of $N_J$ for that upper state.

\section{Excitation Diagram of $\textrm{H}_2$ Level Populations}
We plot the logarithm of $\frac{N_J}{g_J}$, where $g_J$ is the quantum degeneracy level $g_J = g_s(2J + 1)$, as a function of the excitation energy above the ground state ($v = 0$, $J = 0$) of the upper level.

There are two types of $\textrm{H}_2$ depending on whether the two protons are spin aligned or spin anti-aligned. The first type of $\textrm{H}_2$ is ortho-$\textrm{H}_2$. Ortho-$\textrm{H}_2$ protons spin parallel. The second type is para-$\textrm{H}_2$, which describes $\textrm{H}_2$ with protons that spin anti-parallel. For gas with temperatures greater than 200 K and in thermodynamic equilibrium, the ortho-to-para ratio is 3. The value of $g_s$, the nuclear spin statistical weight, changes to keep the wave function of ortho-$\textrm{H}_2$ odd and the wave function of para-$\textrm{H}_2$ even:

\begin{equation} \label{eq:ortho_para}
	g^{ortho}_{J} = 3(2J + 1),\quad g^{para}_{J} = 2J + 1
\end{equation}

In a region of local thermodynamic equilibrium (LTE), the $\textrm{H}_2$ rovibrational level populations will follow the Boltzmann distribution:

\begin{equation} \label{eq:boltzmann}
	\left( \frac{N_J}{g_J} / \frac{N_r}{g_r} \right) \propto \textrm{exp}\left( -\frac{E_J}{kT} \right),\quad \textrm{ln}\left( \frac{N_J}{g_J} / \frac{N_r}{g_r} \right) = -\frac{E_J}{kT}
\end{equation}

\noindent where $k$ is the Boltzmann constant and $T$ is the constant kinetic temperature of the gas. UV excitation of $\textrm{H}_2$ is a non-thermal excitation process that leads to level populations that do not follow the monotonically decreasing slope of the Boltzmann distribution. Instead, $\textrm{H}_2$ excitation resulting from a combination of UV and thermal (collision) processes follows a characteristic ``sawtooth'' pattern (see Figure 1). We have still estimated the $\textrm{H}_2$ level rotation temperatures from the linear fit slope of the level population on the excitation diagram by assuming LTE. We chose to perform this type of fit because our data sets for rovibrational states of constant $v$ greater than $1$ were not information-rich. The derived temperature estimates for each point source are listed in Table 2.

\chapter{Modeling and Interpretation}
\label{chapter:model}

\section{Simulating $\textrm{H}_2$ with Cloudy}
We observe three point sources in the SMC with ARCoIRIS whose bright $\textrm{H}_2$ emission lines arise across up to 9 rovibrational states of constant $v$. We have extracted a line flux data set for each point source sufficient for comparison to model predictions. We use version C17.02 of Cloudy for our models to probe the physics of our regions and determine which $\textrm{H}_2$-rich regions in space they resemble. Cloudy is a plasma simulation code that models interstellar material exposed to an external radiation field and predicts the resultant spectra, line intensities, and column densities. This version of Cloudy includes a robust treatment of $\textrm{H}_2$ that includes the excited electronic and rovibrational states, thermal and non-thermal excitation, photo-dissociation, and formation on icy dust grains.

The physics and chemistry of molecular hydrogen can be modified by the density of gas in the region. We have included the large model of the $\textrm{H}_2$ molecule, described in Shaw et al. (2005), in our Cloudy models to represent several thousand levels. We use the default rates in Cloudy for $\textrm{H}_2$-$\textrm{H}_2$, $\textrm{H}_2$-$\textrm{H}^0$, and $\textrm{H}_2$-$\textrm{He}^0$ collisions from Lee et al. (2008), Wrathmall et al. (2007), and Lee et al. (2006), respectively. For collisional de-excitation rate coefficients where no real data exists, we use the ``g-bar approximation'' to estimate the unobserved collision rate coefficient. We also enabled cosmic-ray flux in all of our Cloudy models. We compare the observed level populations to those for models of the Orion Bar PDR, the PN NGC 7662, and a PPD adapted from the work of Kaplan et al. (2017), Barr\`ia and Kimeswenger (2018), and McClure (2019), respectively. We iterated each initial model 10 times. We then generate a grid of models with constant temperature and density using the model parameters of the initial model that best fits the observed level populations. We describe the three intitial models in Section \ref{sec:models_A} In Section \ref{sec:models_B} and \ref{sec:models_C}, we identify which initial model best fit each point source and present our final best-fit models.

\section{Initial Cloudy Models}
\label{sec:models_A}
\subsection{The Orion Bar PDR Model}
O-type stars are the highest mass stars on the main sequence and have correspondingly short lifetimes. They generally are located in regions of active star formation and have metallicites lower than Solar. The incident radiation field from the O7V star $\omega^1$ Ori C for the Orion Bar PDR model is replicated with parameters taken from the models by Pellegrini et al. (2009) and Shaw et al. (2009). We assumed a static spherical geometry for the model with a constant pressure, temperature, and density. Our constant temperature and density values were taken from the best-fit model of Kaplan et al. (2017). While models that hold these parameters constant do not properly reproduce the structure of the full Orion Bar PDR from the ionization front to the cold molecular regions, the simpler models do reproduce the $\textrm{H}_2$ rovibrational level populations observed within the narrow $\textrm{H}_2$ emitting region. Kaplan et al. (2017) give a full treatment of their model. We present the parameters of their best-fit Cloudy model in Table 2. Parameters that we adapted from other sources are labelled.

\subsection{The NGC 7662 Model}
Observations of the stellar ionizing spectrum of PNe cannot be modeled with blackbody fluxes or non-LTE atmosphere models composed only out of hydrogen and helium. The differences between synthetic spectra and observed spectra at energies greater than 54 eV are rectified by including metal-line blanketing. We used non-LTE, plane-parallel, metal-line blanketed model stellar atmospheres for central stars of PNe (CSPN) created by Rauch (2003) to model NGC 7662. NGC 7662 is a triple-shelled PN with high excitation. We adapted parameters from Barr\`ia and Kimeswenger (2018) to model the non-shocked region of NGC 7662 and avoid modeling a region of highly variable ionisation. We used the element abundances for He, O, N, and S from their best fit model. All other abundances were set to the ISM defaults defined in Cloudy. We enabled ISM grains with a dust-to-gas ratio of 1.5, also adapted from Barr\`ia and Kimeswenger (2018). We used a static spherical geometry for the PN. We list our parameters and abundances for NGC 7662 in Table 3. Parameters adapted from Barr\`ia and Kimeswenger (2018) are labelled, as are parameters from other models of NGC 7662.

\subsection{The A Star PPD Model}
A-type stars are young main sequence dwarfs that mainly emit in the IR with excess beyond contribution from the star alone. The IR excess is emission from a dusty debris disk where planets can form. We modeled the incident radiation field from an average A star with a planet-forming disk around it with parameters adapted from the models by McClure (2019) and Ruaud (2021). We also included an X-ray contribution to the radiation field with a bremsstrahlung emission spectrum so that Cloudy computed physically significant element abundances. McClure used Cloudy to model the inner rim carbon ionization of a PPD around a T Tauri star, while Ruaud computed $\textrm{H}_2$ column densities and resultant disk chemistry using a fiducial disk model. Table 4 lists the parameters of our PPD model, and each parameter is labelled with the author it is attributed to.

\section{Comparison of $\textrm{H}_2$ Level Populations Between Cloudy Models}
\label{sec:models_B}
\subsection{Comparison to the Initial Models}
We leverage the versatility of Cloudy to model three different circumstellar environments and constrain the physics that describes each of our point sources. With our three initial models, we fit our observations of the SMC to each model and evaluate the goodness of fit using a $\chi^2$ parameter of the logarithm of the data-to-model ratios $\Sigma \textrm{log}_{10} ( \textrm{N}_{J} / \textrm{N}_{m} )^2$. $\textrm{N}_{m}$ is the column density of $\textrm{H}_2$ in the relevant energy state calculated by Clody. Our $\chi^2$ method gives each data point an equivalent contribution to the goodness of fit parameter despite the large range in the level populations. The best-fit initial model of each point source is shown in Table 5, along with the goodness of fit, $T$, and $n_H$. We find a minimum agreement within XXX dex between the observed and modeled level populations.

\subsection{Description of the Model Grids}

\section{Fitting the Model Grids}
\label{sec:models_C}

\end{document}
